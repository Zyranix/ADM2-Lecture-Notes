%! TEX root = ../../master.tex
\lecture[Intuition why IP is hard. Big-$\mathcal{O}$ notation]{Di 19 Apr 2022}{Introduction II}


\section{Modelling using IP}

\begin{mini*}{\lambda}{\sum_i \lambda_i f(a_i)}{}{}
    \addConstraint{\sum_{i=1}^{k-1} y_i }{ = 1                }
    \addConstraint{\lambda_i            }{ \leq y_{i-1} + y_i,\quad}{i = 2,\dots,k}
    \addConstraint{\lambda_1            }{ \leq y_1           }
    \addConstraint{\lambda_k            }{ \leq y_{k-1}       }
    \addConstraint{y_i                  }{ \in \mathbb{B}     }
\end{mini*}

\begin{align*}
\end{align*}
This allows $\lambda_{i-1}, \lambda_{i}$ to be positive and rest negative.\footnote{Ch 1: Nemhauser Wolsey}.

Now, given an IP $Q$ could be formulated by \emph{many} $P$'s.
\begin{align*}
    Q:    & \{0000,1000,0100,0010,0110,0101,0011\}                          \\
    P_1 = & \{x \in \realnum^4 \mid 93 x_1 + 49x_2 + 37x_3+29x_4 \leq 111\} \\
    P_2 = & \{x \in \realnum^4 \mid 2 x_1 + x_2 + x_3+ x_4 \leq 2\}         \\
    P_3 = & \{x \in \realnum^4 \mid 2 x_1 + x_2 + x_3+ x_4 \leq 2,          \\
          & x_1 + x_2 \leq 1,                                               \\
          & x_1 + x_3 \leq 1,                                               \\
          & x_1 + x_4 \leq 1\}                                              \\
\end{align*}
Then, $P_3 \subsetneq P_2 \subsetneq P_1$.

\subsection*{Facility location}
Consider following boolean variables:
\begin{align*}
    m: &  & y_i    & = \begin{cases}
        1, & \text{open warehouse } i, \\
        0, & \text{else}
    \end{cases} \\
    n: &  & x_{ij} & = \begin{cases}
        1, & \text{if serve store } j \text{ from warehouse } i, \\
        0, & \text{else}
    \end{cases}
\end{align*}

\subsection*{Complexity}
\emph{Question:} Easy vs. hard?\newline
(some recap of big-$\mathcal{O}$-notation and $\mathcal{P}$ vs. $\mathcal{NP}$)\newline
