%! TEX root = ../../master.tex
\lecture[TODO]{Th 16 June 2022}{Cutting planes}
\todo{content lec16}
\begin{question}
    How do we find Cutting planes?
\end{question}
Consider the Simplex Tableau, and some $x^{\LP} \not \in \integers^n$,
e.g. $x_1^{\LP} \not \in \integers$. Then, $x_1^{\LP} > 0$, and it must be in the basis.
Row 1 of the tableau therefore looks like:
\begin{alignat*}{2}
             &  & x_1 + \overline a_{1,N}^Tx_N                           & = \overline{b}_1    \\
    \implies &  & x_1 + \left\lfloor\overline a_{1,N}^T\right\rfloor x_N & \leq \overline{b}_1
\end{alignat*}
If $x \in P^I$, then the left-hand side is integral, so
\begin{align*}
    x_1 + \left\lfloor\overline a_{1,N}^T\right\rfloor x_N & \leq \left\lfloor\overline{b}_1 \right\rfloor
\end{align*}
and for their difference
\begin{align*}
    f(\overline a_{1,N}^T) x_N & \geq f(\overline{b}_1)
\end{align*}
such that $f$ maps to the fractional part of its input.
This is clearly feasible for all integral vertices in $P$, and thus also for $P^I$.
But $x^{\LP}$ violates it, since the left-hand side is equal to 0, while the right-hand side must be positive -
we found our cutting plane! This particular type is called \vocab[Cutting Plane!Gomory Cut]{Gomory cut}.
\begin{fact}
    Repeatedly finding Gomory cuts leads in a finite number of iterations to an optimal integral solution.
\end{fact}
By 1963 though, Gomory cuts were abandoned, because there were issues that could not be resolved:
\begin{remark}
    Even though they can be easily generated by an optimal simplex tableau, in practice Gomory cuts appear to be slow.
    Additionally, this procedure can be numerically unstable because of rounding errors.
\end{remark}
Maybe we need to think larger - let's try to generalize Gomory Cuts:
\todo{make more clean}
Starting with $Ax \leq b$, we introduce row multipliers $y \in \realnum^m$ such that
\begin{align*}
    y^TAx \leq y^Tb      \\
    y \geq 0             \\
    y^TA \in \integers^n \\
    y^b \text{ fractional}
\end{align*}
Then
\begin{align*}
    y^TAx \leq \left\lfloor y^Tb \right\rfloor
\end{align*}
is a cutting plane, called a \vocab[Cutting Plane!Chv\'atal Cut]{Chv\'atal Cut}, or more general
\vocab[Cutting Plane!Chv\'atal Cut]{Chv\'atal-Gomory Cut}.
\begin{fact}
    The multiplier $y$ is a succinct certificate that this is a valid cutting plane for $P^I$.
    This is also called a \vocab{Chv\'atal proof}.
\end{fact}
\begin{example}
    We try to derive a possible Chv\'atal proof for general matchings. Recall the base-$\LP$
    \begin{maxi*}{x}{w^Tx}{}{}
        \addConstraint{x(\delta(i))}{\leq 1, \quad}{\forall i \in V}
        \addConstraint{x}{\geq 0}
    \end{maxi*}
    which produces non-integral solutions in general, e.g. \autoref{ex:non-int-matching}.
    Therefore, we needed to introduce the blossom constraints for all odd $S \subseteq E$ as cutting planes:
    \begin{align*}
        x(E(S)) \leq \left\lfloor \frac{|S|}{2} \right\rfloor
    \end{align*}
    For Chv\'atal, we now sum the first constraint over all $i \in S$,
    \begin{align*}
        2x(E(S)) + x(\delta(S)) \leq |S|,
    \end{align*}
    and sum the second constraint multplied by $-\frac{1}{2}$ for all $e \in \delta(S)$
    \begin{align*}
        -x(\delta(S)) \leq 0,
    \end{align*}
    In total and divided by 2, we get
    \begin{align*}
        x(E(S)) \leq \frac{|S|}{2}.
    \end{align*}
    Because all properties hold, we can round down and get our Chv\'atal plane.
    While in this case these CPs are already enough to cut down to $P^I$, in general it usually is more complicated.
\end{example}
Let's find some properties about Chv\'atal-Gomory cuts.
\begin{theorem}
    Define $P^\prime$ as $P$ with \emph{all} possible Chv\'atal-Gomory cuts added.
    Then $P^\prime$ is a polyhedron.
\end{theorem}
\todo{tikz graphic}
\begin{proof}
    As a general idea, we will show that every Chv\'atal cut is equivalent to a Chv\'atal cut
    generated from a finite set $S$. Then, it follows directly that $P^\prime$ is still a polyhedron.

    For $S$, we define
    \begin{align*}
        S \coloneqq \{x \mid \exists 0 \leq y < 1: x = y^TA \AND y^TA \in \integers^n\}
    \end{align*}
    which is obviously finite.

    Suppose there is a \enquote{big} $\overline y$ that defines any Chv\'atal cut via
    \begin{align*}
        (\overline y^T A)x & \leq \left\lfloor \overline yb \right\rfloor.
    \end{align*}
    We define
    \begin{align*}
        \tilde y \coloneqq f(\overline y).
    \end{align*}
    Note $0 \leq \tilde y < 1$. Calculations show:
    \begin{align*}
        \tilde y^TA = \underbrace{\overline y^TA}_{\text{integral}} - \underbrace{\left\lfloor \overline y \right\rfloor A}_{\text{integral}}
    \end{align*}
    which implies $\tilde y \in S$.
    Also,
    \begin{align*}
        \tilde y^Tb = \overline y^Tb - \underbrace{\left\lfloor \overline y \right\rfloor b}_{\text{integral}}
    \end{align*}
    So, $\tilde y^Tb$ and $\overline y^Tb$ have the same fractional part, or
    \begin{align*}
        \left\lfloor \tilde y b\right\rfloor & = \left\lfloor \overline y \right\rfloor b = \left\lfloor \overline y b\right\rfloor
    \end{align*}
    \todo{finish proof}

    See also \cite[Thm.~6.34]{comb-optimization-cook}.
\end{proof}
\begin{theorem}
    There is an $k \in \natnum $ such that repeatedly calculating Chv\'atal closure $P_{i+1}$ from $P_i$ (that is, $P_{i+1} \coloneqq P_i^\prime$)
    will lead to $P^I = P_k$. We call this $k$ the \vocab{Chv\'atal rank} of $P_1$.
\end{theorem}