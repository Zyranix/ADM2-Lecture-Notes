%! TEX root = ../../master.tex
\lecture[$\coNP$. Ellipsoid method. Separation vs. optimization.]{Di 03 May 2022}{Ellipsoid method}
\todo{Content lec06}
\begin{theorem}[Minkowski]
    For a polyhedron $P$ it holds $x \in P$ iff there exist vertices $v_1,...,v_k$
    and rays $r_1, ..., r_l$, such that
    \begin{align*}
        \sum_i \lambda v_i + \sum_j \mu_j r_j & = x    \\
        \sum_i \lambda_i                      & = 1    \\
        \lambda, \mu                          & \geq 0
    \end{align*}
\end{theorem}
\begin{proof}
    See ADM1.
\end{proof}
\todo{add ref HW?}
\begin{conclusion}
    Depending on the representation we have, we have different ways to solve $\OPT$ and $\SEP$:

    \begin{tabular}{lll}
                                    & Hull repr.                                & Vertex-repr                        \\ \cline{2-3}
        \multicolumn{1}{l|}{$\OPT$} & \multicolumn{1}{l|}{LP Simplex/Ellipsoid} & \multicolumn{1}{l|}{Brute Force}   \\ \cline{2-3}
        \multicolumn{1}{l|}{$\SEP$} & \multicolumn{1}{l|}{Brute Force}          & \multicolumn{1}{l|}{LP (Homework)} \\ \cline{2-3}
    \end{tabular}
\end{conclusion}
\todo{picture example}
\begin{example}
    Consider the $n$-cube $C^n \coloneqq \{x \in \realnum^n \mid -1 \leq x_i \leq 1\}$.
    It has $2n$ facets, but $2^n$ vertices.

    Now, consider the polar of $C^n$, which can be shown to be the $n$-octahedron $O^n$.
    Remember the intuition, that the polar exchanges vertices with facets. Indeed it holds that
    now, we have $2^n$ facets, but only $2n$ vertices.
\end{example}
\todo{hyperlink polymake}
\begin{info}
    Polymake is a tool for converting between H-representation and V-representation.
\end{info}
\begin{question}
    Consider the problem of finding a solution $x$ to $Ax=b$.
    How do we construct succinct certificates of feasibility and infeasibility?
\end{question}
\begin{theorem}
    Exactly one of the following systems is feasible:
    \begin{align*}
        Ax & =b & \text{vs.} &  & y^TA & =0 \\
           &    &            &  & y^Tb & =1 \\
    \end{align*}
\end{theorem}
\begin{proof}
    Suppose both are feasible. Then we have solutions $y^0,x^0$, and can construct following contradiction:
    \begin{align*}
                        &  & Ax^0                        & = b            \\
        \Leftrightarrow &  & \underbrace{(y^0)^TA}_0 x^0 & = (y^0)^Tb = 1
    \end{align*}
    It remains to proove at least one system is feasible. We can use \vocab{Gaussian Elimination} for that:
    Gaussian Elimination either yields a solution $x^0$ we can use as a succinct certificate for feasibility,
    or determine it is infeasible by yielding the row multiplier $y^0$ as a succinct certificate of infeasibility.
\end{proof}
\begin{strategy}
    The method we used in previous proof is called \vocab{Theorem of the Alternative}.
\end{strategy}
\begin{question}
    Now consider the problem of finding a solution $x$ to $Ax \leq b$.
    How do we construct succinct certificates of feasibility and infeasibility?
\end{question}
\begin{theorem}[Farkas Lemma]
    \label{thm:farkas}
    Exactly one of the following systems is feasible:
    \begin{align*}
        Ax & \leq b & \text{vs.} &  & y^TA & =0     \\
           &        &            &  & y^Tb & <0     \\
           &        &            &  & y    & \geq 0 \\
    \end{align*}
    This is also known as \vocab{Farkas Lemma}.
\end{theorem}
\begin{proof}
    Suppose both are feasible. Analoguous to previous proof we can see the contradiction:
    \begin{align*}
                        &  & Ax^0                        & \leq b            \\
        \Leftrightarrow &  & \underbrace{(y^0)^TA}_0 x^0 & \leq (y^0)^Tb < 0
    \end{align*}
    At least one system is feasible, which we can see by using Phase 1 of the Simplex Algorithm, and the Ellipsoid Method,
    which can generate certificates $x$ or else $y$.
\end{proof}
\begin{definition}[Diophantine equations]
    An equation of the form $Ax=b$, for $x \in \integers^n$, is called \vocab{diophantine equation}.
\end{definition}
\begin{theorem}
    Exactly one of following is feasible:
    \begin{align*}
        Ax & \leq b          & \text{vs.} &  & y^TA & \in \integers^n    \\
        x  & \in \integers^n &            &  & y^Tb & \not \in \integers \\
    \end{align*}
\end{theorem}
\begin{proof}
    Suppose both are feasible. Then
    \begin{align*}
                        &  & Ax^0                                                               & = b                                          \\
        \Leftrightarrow &  & \underbrace{(y^0)^TA}_{\integers^n} \underbrace{x^0}_{\integers^n} & = \underbrace{(y^0)^Tb}_{\not \in \integers}
    \end{align*}
    We can use the \vocab{Hermite Normal Form} algorithm to show that at least one system is feasible.
    Note that HNF is a polynomial algorithm.
\end{proof}
\begin{conclusion}
    Summing everything up for feasibility of linear systems:

    \begin{tabular}{lll}
                                    & continuous                & integer                           \\ \cline{2-3}
        \multicolumn{1}{l|}{$=$}    & \multicolumn{1}{l|}{G.E.} & \multicolumn{1}{l|}{HNF}          \\ \cline{2-3}
        \multicolumn{1}{l|}{$\leq$} & \multicolumn{1}{l|}{LP}   & \multicolumn{1}{l|}{not possible} \\ \cline{2-3}
    \end{tabular}

    The problem with integer inequality systems is missing duality, e.g. there is no way of generating succinct certificates for verifying infeasibility, making
    it impossible to use the Theorem of the Alternative.
\end{conclusion}
Another usage of Theorem of the Alternative:
\begin{align*}
    Ax & = b    & \Leftrightarrow                        &  & Ax                            & \leq b       \\
    x  & \geq b &                                        &  & -Ax                           & \leq -b      \\
       &        &                                        &  & -x                            & \leq 0       \\
    \\
       &        & \overset{\ref{thm:farkas}}{\text{vs.}} &  & (y^1)^TA - (y^2)^TA - (y^3)^T & = 0          \\
       &        &                                        &  & (y^1)^Tb - (y^2)^Tb           & <0           \\
       &        &                                        &  & y^1,y^2,y^3                   & \geq 0       \\
    \\
       &        & \Leftrightarrow                        &  & y^TA                          & \geq 0       \\
       &        &                                        &  & y^Tb                          & < 0          \\
       &        &                                        &  & y                             & \text{ free} \\
\end{align*}
\begin{theorem}[Gourdan]
    Consider $Ax < 0$.
\end{theorem}
\todo{write Gourdan}
Consider an LP with lower and upper bounds:
\begin{align*}
    \min \quad c^Tx \\
    Ax=b            \\
    l \leq x \leq u
\end{align*}
Decompose:
\begin{align*}
    \min \quad c^Tx \\
    Ax=b            \\
    x \geq l        \\
    -x \geq u       \\
    x \text{ free}
\end{align*}
Dualize:
\begin{align*}
    \max \quad b^Ty + l^T\lambda - u^T \mu \\
    y^TA + \lambda^T - \mu^T = c^T         \\
    \lambda,\mu \geq 0                     \\
    y \text{ free}
\end{align*}
Rewrite