%! TEX root = ../../master.tex
\lecture[Definitions for ILP. Binary LP. Disaggregation.]{Tu 19 Apr 2022}{Introduction}
% \begin{orga}
%     \begin{itemize}
%         \item    Die Vorlesung wird aufgezeichnet.
%         \item Wir duzen uns.
%         \item Für die Übungen muss man sich auf eCampus anmelden, ob Do, 20:00 Uhr (Th 15 Apr 2021 20:00 Uhr)
%         \item Die Übungsblätter werden Donnerstag zur Verfügung gestellt und werden nach 10 Tagen am Montag, 10 Uhr abgegeben.
%         \item Es wird eine Fragestunde um Donnerstag, 16 Uhr geben.
%         \item Es wird kein Skript geben, allerdings werden die geschriebenen Notizen auf eCampus zur Verfügung gestellt.
%         \item Die Vorlesung orientiert sich an der vom letzten Jahr.
%               % \item Für Literatur sind empfohlen: \cite{topology-waldhausen}, \cite{algebraic-topology-hatcher} sowie \cite{topology-and-geometry} (auch auf der Vorlesungshomepage zu finden).
%     \end{itemize}
% \end{orga}

\section{Introduction}
In ADM1 we often already worked with Integer Programming and just assumed everything is fine.
In this course however, we want to find out how Integer Programming actually works, why it is generally "hard", and under which circumstances it is "easy".
\begin{definition}[Flavors of $\IP$]
    First of all, we want to define different variants of \vocab{Integer Programming}:
    \begin{itemize}
        \item \vocab[Integer Programming!pure]{Pure Integer Programming} assumes \emph{all} variables are integer.
        \item \vocab[Integer Programming!mixed]{Mixed Integer Programming} also allows some variables to be real.
        \item \vocab[Integer Programming!binary]{Binary Integer Programming}, also called 0-1-Integer-Programming, restricts the integer variables to $\bool \coloneqq\{0,1\}$. Mixed variants are also possible.
    \end{itemize}
\end{definition}
\begin{question}
    Why is $\IP$ harder than $\LP$?
    Naively, one would assume this should \emph{not} be the case, because our search space is smaller (at most countably infinite)!
\end{question}
Let's solve the $\IP$ in ADM1-style - suppose
\begin{maxi*}{x}{w^Tx}{}{}
    \addConstraint{x \in Q}{ = \{x \in \integers^n \colon Ax \leq b\}}
\end{maxi*}
For simplicity, assume $Q$ is bounded. Then the set of feasible points in $Q$
is finite, and therefore we can consider the polytope
\begin{align*}
    \conv(Q) = \{x \in \realnum^n \mid A'x \leq b'\}
\end{align*}
for suitable $A'$ and $b'$. Notice all vertices must be in $Q$ and thus are integral.
As a consequence, it is sufficient to solve the $\LP$
\begin{maxi*}{x}{w^Tx}{}{}
    \addConstraint{A'x}{\leq b'.}
\end{maxi*}
\begin{warning}
    Computing $A',b'$ is non-trivial! In fact, computing the \vocab{integer hull} $\conv(Q)$
    is what makes $\IP$ hard.
\end{warning}

\subsection{$\IP$ is "hard"}
We can gather more evidence that $\IP$ must be hard.
\begin{theorem}
    \label{thm:ip-logical-decomposition}
    Every logical statement can be expressed with integer programming.
\end{theorem}
\begin{proof}
    It suffices to show that for variables $x_1,x_2,y \in \bool$ we can find $\IP$s that model $\AND, \OR, \NOT$.
    \begin{itemize}
        \item $y = x_1 \AND x_2$ can be modeled as
              \begin{align*}
                  y & \leq x_1           \\
                  y & \leq x_2           \\
                  y & \geq x_1 + x_2 - 1
              \end{align*}
        \item $y \coloneqq x_1 \OR x_2$ can be modeled as
              \begin{align*}
                  y & \geq x_1       \\
                  y & \geq x_2       \\
                  y & \leq x_1 + x_2
              \end{align*}
        \item $y \coloneqq \neg x$ can be modeled as
              \begin{align*}
                  y & = 1- x
              \end{align*}
    \end{itemize}
\end{proof}
\begin{theorem}
    \label{thm:ip-model-union}
    $\IP$ can model (finite) unions of polyhedra, e.g. non-convex problems.
\end{theorem}
\begin{proof}
    Consider
    \begin{align*}
        P \coloneqq \bigcup_{i=1}^k \underbrace{\{x \mid A_ix \leq b_i\}}_{P_i}
    \end{align*}
    and introduce auxiliary binary variables
    \begin{align*}
        y_i \coloneqq \begin{cases}
                          1, & \text{if $x \in P_i$},   \\
                          0, & \text{if we don't care.} \\
                      \end{cases}
    \end{align*}
    Now, assume $M \in \realnum$ large enough (\vocab[Big-M method]{Big-$M$ method}), such that
    \begin{align*}
        P_i \subset \overline{P} \coloneqq \{x \mid A_ix \leq b_i + M \cdot \mathbf{1}\}
    \end{align*}
    for all $i$. Thus, we can construct following $\IP$:
    \begin{mini*}{x}{w^Tx}{}{}
        \addConstraint{A_ix}{\leq b_i + (1-y_i) M \cdot \mathbf{1}, \quad}{i=1, \dots, k}
        \addConstraint{\sum_i y_i}{=1}
    \end{mini*}
    This forces exactly one $y_i$ to $1$, resulting that $x \in P_i$ and $x \in \overline{P}$ is sufficient.
    We call this method \vocab[Big-M method!righthandside]{righthandside Big-$M$}, short RHS.
    Further information can be found in \cite[Ch. 1]{int-comb-optimization}.
\end{proof}

\begin{note}
    It's also possible to handle $M$ as a symbolic value, but this makes other things more complicated.
\end{note}

\begin{problem}
Finding $M$ big enough can make $\LP$ hard to solve, because of matrix-inversions getting numerically unstable.
\end{problem}

\begin{proof}[Alternative proof of \autoref{thm:ip-model-union}]
    Assume that we can bound each $x \in P_i$ by $u_i$, i.e. $x \leq u_i$  (note that this is basically a hidden big-$M$!).
    Now we can disaggregate $x$ for each $P_i$ as its own private $x_i \in \realnum^k$, and analogously introduce $y_i \in \bool$ to restrict ourselves to one polyhedron:
    \begin{mini*}{x}{w^Tx}{}{}
        \addConstraint{A_ix_i}{\leq y_ib_i, \quad}{i=1,\dots,k}
        \addConstraint{x_i}{\leq y_iu_i, \quad}{i=1,\dots,k}
        \addConstraint{\sum_{i=1}^n y_i}{=1}
        \addConstraint{\sum_{i=1}^n x_i}{=x}
        \addConstraint{x,x_i}{\geq 0}
    \end{mini*}
    Again, exactly one $y_i$ is equal to $1$, forcing the other $x_j$ to be equal to $0$,
    and thus setting $x$ to $x_i$.
    We call this method \vocab[Big-M method!upper bound]{upper bound on $x$ Big-$M$}, short UBX.
\end{proof}
\begin{remark}
    In general, it cannot be said if RHS or UBX is better.
    Even though RHS only introduces $n$ new variables as opposed to UBX's $nk$ variables, UBX's disaggregated
    formulation often is \emph{tighter} in the sense that the $\LP$ relaxation is closer to the integer hull.
\end{remark}