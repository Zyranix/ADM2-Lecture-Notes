%! TEX root = ../../master.tex
\lecture[TODO]{Tu 28 June 2022}{TODO}
\todo{content lec18}

\begin{conclusion}
    The Chv\'atal rank is \emph{always} finite.
\end{conclusion}
\begin{remark}
    Nonetheless, the Chv\'atal rank can be arbitrarily large. Consider:
    \vspace{5pt}
    \\
    \begin{minipage}{\textwidth}
        \centering
        \begin{tikzpicture}
            \begin{scope}[every node/.style={circle}]

                \node[anchor=north] (0) at (1,0) {$(0,0)$};
                \node[anchor=north] (1) at (3,0) {$(1,0)$};
                \node[anchor=south] (C) at (2,3) {$(\frac{1}{2},k)$};
                \coordinate (S) at (0,0);
                \coordinate (E) at (4,0);

            \end{scope}
            \path (0) edge (C);
            \path (C) edge (1);
            \path (S) edge (E);
        \end{tikzpicture}
        % \captionof{figure}{A graph with $S$ colored orange}
    \end{minipage}
    Here, the Chv\'atal rank is $2k$.
\end{remark}
We can also give similar qualitative measures for cuts:
\begin{definition}
    If the cutting plane $\alpha^Tx \leq \beta$ is valid for $P^k$, but not $P^{k-1}$, then this cutting plane
    has \vocab{Chv\'atal} rank $k$.

    Furthermore, we say the cutting plane is $\vocab{facet-inducing}$ for $P^I$ if the CP contains a facet of $P^I$.
\end{definition}
There are several methods of finding facet-inducing cutting planes:
\begin{enumerate}
    \item Find $\bar x$ such that $\alpha^T\bar x > \beta$, but $\bar x$ satisfies every other constraint of $P^I$.
          However, we don't know $P^I$.
    \item Find feasible, affinely independent $v^1,\dots,v^n \in P^I$ such that for all $i$ it holds $\alpha^Tv^i = \beta$
    \item ..
\end{enumerate}
It is simpler when $P^I$ is full-dimensional, i.e. $\dim(P^I)=n$.

It remains to show how we can solve $\MIP$s. Recall the fundamental theorem of $\MIP$ \autoref{thm:fundamental_mip}.
Consider following simple-looking $\MIP$ for some fractional $b$:
\begin{align*}
    x + y & \geq b                 \\
    x     & \in \realnum_{\geq 0}  \\
    y     & \in \integers_{\geq 0}
\end{align*}
How can we cut off non-integer points without violating the continuity of $x$?
Define a line
\begin{align*}
    l & = \{(x,y) \mid x + f(b)y \geq f(b)\left\lceil b \right\rceil \}
\end{align*}
We see this yields our integer hull.
What about the general case, that is, $x \in \realnum_{\geq 0}^n, y \in \integers_{\geq 0}^p$?
\begin{align*}
                    &  & \sum_j a_jx_j + \sum_k d_ky_k                                                                                                                                                                                                   & \geq b                               \\
    \Leftrightarrow &  & \left(\underbrace{\sum_{j:a_j \leq 0} a_jx_j}_{\leq 0} + \sum_{j:a_j > 0} a_jx_j\right) + \left( \sum_k \left\lfloor d_k \right\rfloor y_k + \sum_{k : f(d_k) < f(b)} f(d_k)y_k + \sum_{k : f(d_k) \geq f(b)} f(d_k)y_k \right) & \geq b                               \\
    \Leftrightarrow &  & \underbrace{\sum_{j:a_j > 0} a_jx_j + \sum_{k : f(d_k) < f(b)} f(d_k)y_k}_{\in \realnum_{\geq 0}} + \underbrace{\sum_k \left\lfloor d_k \right\rfloor y_k +  \sum_{k : f(d_k) \geq f(b)} y_k}_{\in \integers_{\geq 0}}          & \geq b                               \\
    \Leftrightarrow &  & \sum_{j:a_j > 0} a_jx_j + \sum_{k : f(d_k) < f(b)} f(d_k)y_k + f(b) \sum_k \left\lfloor d_k \right\rfloor y_k +  \sum_{k : f(d_k) \geq f(b)} y_k                                                                                & \geq f(b) \left\lceil b \right\rceil
\end{align*}
We call this \vocab{Gomory Mixed Integer Cuts}, shorthand $\GMI$, and \vocab{Mixed Integer Rounding}, shorthand MIR.
Consider
\begin{align*}
    \sum_j a_j^+x_j + \sum_k \min(f(d_k), f(b))y_k + f(b) \sum_k \left\lfloor d_k \right\rfloor y_k \geq f(b) \left\lceil b \right\rceil.
\end{align*}
We introduce a technique to find $\GMI$ cuts.
Consider row multipliers $\lambda \geq 0$ on
\begin{alignat*}{2}
    \lambda: &  & Ax+Dy & \geq b
\end{alignat*}{2}
and slack variables
\begin{align*}
    Ax+Dy - Is = b
\end{align*}
so that we get
\begin{align*}
    (\lambda^TA)x+(\lambda^TD)y - \lambda^Ts = \lambda^Tb
\end{align*}
with $\lambda$ now being free. Applying $\GMI$ yields
\begin{equation*}
    \begin{split}
        \sum_j (\lambda^T A)^+_j x_j + f(\lambda^Tb)\sum_k \left\lfloor (\lambda^T D)_k\right\rfloor y_k + \quad &\\
        + \sum_k \min(f(\lambda^TD_k),f(\lambda^Tb))y_k + \sum_{\lambda_i <0} |\lambda_i|s_i &\geq f(\lambda^Tb)\left\lceil \lambda^Tb \right\rceil
    \end{split}
\end{equation*}
Now, substitute $s \coloneqq Ax + Dy - b$.
\todo{example...}
\begin{conclusion}
    $\GMI$ cuts can be stronger than Chv\'atal-Gomory cuts.
\end{conclusion}
Consider a disjunction $y \leq \beta \OR y\geq \beta + 1$.
We can find a \vocab{Split Cut} as a generalization of $\CG$ and $\GMI$ cuts.
\todo{examples...}
\begin{theorem}
    Define the polyhedron generated by adding all $\GMI$ (split) cuts as $P^{\GMI}$ ($P^{\mathsf{split}}$).
    Then $P^{\mathsf{split}} = P^{\GMI} \subset P^1 = P^{\CG}$.
\end{theorem}